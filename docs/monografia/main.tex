\documentclass{beamer}

% -------- PREÁMBULO --------
\usepackage[T1]{fontenc}
\usepackage[utf8]{inputenc}
\usepackage[spanish]{babel}
\usepackage{csquotes}
\usepackage{graphicx}
\usepackage{hyperref}
\usepackage{tikz}
\usetikzlibrary{positioning}
\usepackage{pgffor}
\usepackage{comment}
\usepackage{xurl}
\usepackage{multicol}
\usepackage[table]{xcolor}


% --- Bibliografía con biblatex (APA en español, pero con corchetes) ---
\usepackage[backend=biber,style=apa]{biblatex}
\DeclareLanguageMapping{spanish}{spanish-apa}
\addbibresource{referencias.bib}

% --- Redefinir comandos para usar corchetes en las citas ---
\renewcommand{\parencite}[1]{[\citeauthor{#1}, \citeyear{#1}]}
\renewcommand{\cite}[1]{\parencite{#1}}

% --- Ajuste global del tamaño de las referencias finales ---
\renewcommand*{\bibfont}{\scriptsize}

% Tema y estilo
\usetheme{Madrid}
\setbeamertemplate{caption}[numbered]

% -------- Colores --------
\definecolor{unalazul}{RGB}{0,56,168} % Azul institucional UNAL

% Aplicar color institucional al encabezado
\setbeamercolor{frametitle}{bg=unalazul, fg=white}
\setbeamercolor{section in head/foot}{bg=unalazul, fg=white}
\setbeamerfont{section in head/foot}{size=\scriptsize}

% -------- Pie de página --------
\setbeamertemplate{footline}{%
  \leavevmode%
  \hbox{%
    \begin{beamercolorbox}[wd=.45\paperwidth,ht=2.5ex,dp=1.5ex,center]{footline left}
      \color{white}\scriptsize Edgar Iván Calpa (UNAL Manizales)
    \end{beamercolorbox}%
    \begin{beamercolorbox}[wd=.30\paperwidth,ht=2.5ex,dp=1.5ex,center]{footline center}
      \color{white}\scriptsize EEGNet y Shallow ConvNet
    \end{beamercolorbox}%
    \begin{beamercolorbox}[wd=.25\paperwidth,ht=2.5ex,dp=1.5ex,center]{footline right}
      \color{white}\scriptsize \insertframenumber/\inserttotalframenumber
    \end{beamercolorbox}%
  }
  \vskip0pt%
}
\setbeamercolor{footline left}{bg=unalazul}
\setbeamercolor{footline center}{bg=unalazul}
\setbeamercolor{footline right}{bg=unalazul}

% -------- Encabezado personalizado --------
\newcommand{\customframetitle}{%
  \nointerlineskip%
  \begin{beamercolorbox}[wd=\paperwidth,ht=1.5ex,dp=0ex]{section in head/foot}%
    \hskip-0.3ex
    \vbox to 0.8ex{\vfil\scalebox{0.75}{\tiny\insertsectionnavigationhorizontal{\paperwidth}{}{}}}\vss%
  \end{beamercolorbox}%
  \vskip-0.3pt%
  \nointerlineskip%
  \begin{beamercolorbox}[wd=\paperwidth,ht=2ex,dp=1ex]{frametitle}%
    \vspace{-0.3ex}
    \usebeamerfont{frametitle}\insertframetitle%
    \hfill%
  \end{beamercolorbox}%
  \begin{tikzpicture}[remember picture,overlay]
    \node[anchor=north east, xshift=0.1cm, yshift=0.1cm] at (current page.north east) {%
      \includegraphics[width=0.14\paperwidth]{figuras/image_reducido.png}%
    };
  \end{tikzpicture}%
}

\newcommand{\frametitleOutline}{%
  \begin{tikzpicture}[remember picture,overlay]
    \fill[unalazul] (current page.north west) rectangle ([yshift=-5ex]current page.north east);
  \end{tikzpicture}
  \begin{beamercolorbox}[wd=\paperwidth,ht=1ex,dp=1ex]{frametitle}%
    \usebeamerfont{frametitle}\insertframetitle%
  \end{beamercolorbox}%
  \begin{tikzpicture}[remember picture,overlay]
    \node[anchor=north east, xshift=0.1cm, yshift=-0.2cm] at (current page.north east) {%
      \includegraphics[width=0.14\paperwidth]{figuras/image_reducido.png}%
    };
  \end{tikzpicture}%
}

% -------- METADATOS --------
\title{EEGNet y Shallow ConvNet}
\author{Universidad Nacional de Colombia - Sede Manizales \\ Edgar Iván Calpa Cuacialpud}
\date{Modelos de aprendizaje profundo aplicados a señales EEG\\% 
Profesor: Andrés Marino Álvarez Meza, PhD\\%
Departamento de Ingeniería Eléctrica, Electrónica y Computación}

\begin{document}

% ---- Portada ----
\begin{frame}[plain]
  \titlepage
  \begin{center}
    \includegraphics[width=0.3\linewidth]{figuras/Unalescudo_reducido.png}
  \end{center}
\end{frame}


% ---- Outline ----
\setbeamertemplate{frametitle}{\frametitleOutline}
\begin{frame}{Outline}
  \scriptsize   % o \tiny si necesitas aún más pequeño
  \tableofcontents
\end{frame}


% Activar la barra de navegación y el escudo a partir de aquí (diapositiva 3 en adelante)
\setbeamertemplate{frametitle}{\customframetitle}

% ---- Motivation ----
\section{Motivation}
\begin{frame}{Motivation}
\begin{columns}[T]

  % Columna izquierda: lista de motivaciones
  \begin{column}{0.6\textwidth}
    \begin{itemize}
      \item \textbf{Accesibilidad clínica}: PET tau y LCR son costosos e invasivos, se requiere biomarcador no invasivo y repetible {\tiny\cite{dePaula2009}}
      \begin{figure}
      \centering
      \href{https://www.savalnet.cl/medios/cyc/articulos/2024/728546cg01}{%
        \includegraphics[width=0.3\linewidth]{figuras/tau.png}%
      }
      \caption{\tiny Prueba sanguínea de tau.}
    \end{figure}
      \item \textbf{Huella funcional}: La propagación de tau altera ritmos y conectividad neuronal, el EEG puede capturar esos cambios {\tiny\cite{Tolnay1999}}
    \end{itemize}
  \end{column}

  % Columna derecha: imágenes con enlaces
  \begin{column}{0.38\textwidth}

    % Imagen 2: Huella funcional
    \begin{figure}
      \centering
      \href{https://www.nature.com/articles/nn.4328}{%
        \includegraphics[width=\linewidth]{figuras/Performance_of_ESM_in_predicting_spatial_progression_of_tau.png}%
      }
      \caption{\tiny Propagación de tau y conectividad neuronal.}
    \end{figure}
  \end{column}

\end{columns}
\end{frame}

% ---- Motivation ----
%\section{Motivation}
\begin{frame}{Motivation}
\begin{columns}[T]

  % Columna izquierda: lista de motivaciones
  \begin{column}{0.6\textwidth}
    \begin{itemize}
      \item \textbf{Deep learning en EEG}: Modelos compactos como EEGNet y Shallow ConvNet detectan patrones espacio–temporales en EEG con datos limitados {\tiny\cite{Lawhern2018,Schirrmeister2017}}
    \end{itemize}
  \end{column}

  % Columna derecha: imágenes con enlaces
  \begin{column}{0.4\textwidth}
    % Imagen 3: Deep learning en EEG
    \begin{figure}
      \centering
      \href{https://www.mdpi.com/2079-9292/12/12/2743}{%
        \includegraphics[width=\linewidth]{figuras/EEGNet.png}%
      }
      \caption{\tiny Arquitectura EEGNet.}
    \end{figure}
  \end{column}

\end{columns}
\end{frame}


% ---- Problem Statement ----
\section{Problem Statement}
\begin{frame}{Problem Statement}
\begin{itemize}
  \item \textbf{General}: No existe un biomarcador funcional, no invasivo y accesible para detectar progresión de Alzheimer vinculada a tau {\tiny\cite{dePaula2009}}
  \item \textbf{Específico}: Validar si EEGNet y Shallow ConvNet identifican patrones EEG asociados a propagación de tau y progresión clínica {\tiny\cite{Lawhern2018,Schirrmeister2017}}
  \item \textbf{Qué hacemos}: Entrenar y comparar ambos modelos sobre EEG CN/MCI/AD, analizar interpretabilidad por canal/banda y vincular hallazgos con trayectorias tau {\tiny\cite{Ajra2023}}
\end{itemize}
\end{frame}

% ---- Problem Statement ----
% \section{Problem Statement}
\begin{frame}{Problem Statement}
\begin{itemize}
  \item \textbf{Teoría de detección}: Maximizar \(P_{D}\) para un \(P_{FA}\) dado, evaluar curvas ROC y umbrales {\tiny\cite{Kay1998}}
  \item \textbf{Neurobiología de tau}: Hiperfosforilación, ovillos neurofibrilares y patrón hipocampo→entorhinal→cortezas asociativas {\tiny\cite{Tolnay1999}}
  \item \textbf{Implicación EEG}: Cambios en alfa/beta/gamma y sincronización funcional reflejan la disrupción de redes por tau {\tiny\cite{dePaula2009}}
\end{itemize}
\end{frame}

% ---- State of the art ----
\section{State of the art}
\begin{frame}{State of the art}
\scriptsize
\renewcommand{\arraystretch}{0.8} % Ajusta el espacio entre filas si es necesario
\vspace{-0.8cm}
\begin{table}
\centering
\begin{tabular}{p{2.6cm} p{2.6cm} p{2.6cm} p{2.6cm}}
\textbf{Enfoque / Recurso} & \textbf{Características clave} & \textbf{Ventajas} & \textbf{Limitaciones} \\
\hline
EEGNet {\tiny\cite{Lawhern2018}} & CNN compacta con convoluciones separables & Generaliza con pocos datos & Interpretabilidad limitada por canal/banda \\
\hline
Shallow ConvNet {\tiny\cite{Schirrmeister2017}} & Arquitectura superficial para ritmos mu/beta & Baseline reproducible en BCI & Sensible a ruido y variabilidad intersujeto \\
\hline
CNN + AEC/PLV {\tiny\cite{Ajra2023}} & Conectividad funcional para AD vs FTD & Alta exactitud en clasificación clínica & No explora biomarcadores tau directamente \\
\hline
Biomarcadores tau {\tiny\cite{dePaula2009}} & PET y LCR como gold standard & Alta especificidad & Costosos e invasivos \\
\hline
Dataset clínico Alzheimer/FTD {\tiny\cite{MDPI2023}} & EEG clínico en reposo, 88 sujetos (AD, FTD, controles) & Datos reales, formato BIDS, accesible & Tamaño limitado, variabilidad clínica \\
\hline
BCI Competition IV {\tiny\cite{Blankertz2008}} & Datasets estandarizados de EEG/MEG/ECoG para clasificación & Benchmark internacional, comparabilidad de algoritmos & No enfocado en demencia, contexto experimental \\
\hline
\end{tabular}
\end{table}

\vspace{-0.3cm}
\textbf{Brecha actual:} Falta conexión directa entre modelos deep-EEG y biomarcadores tau funcionales. Se requiere validación clínica con interpretabilidad por canal/banda.
\end{frame}



% ---- Novelty ----
% \section{Novelty}
\begin{frame}{Novelty}
\begin{itemize}
  \item \textbf{Innovación}: Integrar neurobiología de tau con detección funcional en EEG mediante CNN compactas como EEGNet y ShallowConvNet {\tiny\cite{Ajra2023,Lawhern2018}}
  \item \textbf{Etiqueta del estudio}: Propuesta de biomarcador EEG funcional no invasivo para progresión clínica de Alzheimer
  \item \textbf{Evaluación}: Accuracy, F1 macro, AUC ROC, \( P_D \), \( P_{FA} \); mapas de importancia por canal y banda {\tiny\cite{Kay1998}}
  \item \textbf{Conexión}: El pipeline propuesto permite vincular patrones EEG con trayectorias de tau, reforzando la hipótesis funcional
\end{itemize}
\end{frame}

% ---- Objectives ----
\section{Objectives}
\begin{frame}{Objectives}
\begin{itemize}
  \item \textbf{General}: Desarrollar y validar un modelo EEG (EEGNet vs Shallow) para detectar patrones asociados a tau y progresión de Alzheimer {\tiny\cite{Lawhern2018,Schirrmeister2017}}
  \item \textbf{Específicos}:
    \begin{itemize}
      \item Preprocesar EEG CN/MCI/AD y segmentar ventanas 2–4 s {\tiny\cite{Schirrmeister2017}}
      \item Entrenar y comparar EEGNet vs Shallow ConvNet en clasificación CN vs AD y CN vs MCI {\tiny\cite{Lawhern2018}}
      \item Interpretabilidad: analizar bandas y canales relevantes y su coincidencia con trayectorias tau {\tiny\cite{Ajra2023}}
      \item Validar con severidad clínica (MMSE/MoCA) y, si hay, PET/LCR tau {\tiny\cite{dePaula2009}}
    \end{itemize}
\end{itemize}
\end{frame}


% ---- Experimental design ----
\section{Methodology}
\begin{frame}{Experimental design}
\begin{itemize}
  \item \textbf{Datos}: EEG multicanal CN/MCI/AD con metadatos clínicos; subcohorte PET/LCR si está disponible {\tiny\cite{Ajra2023}}
  \item \textbf{Entradas}: Señales EEG (C×T), ventanas 2–4 s, normalización; opcional espectrogramas y matrices PLV/PLI {\tiny\cite{Lawhern2018,Schirrmeister2017}}
  \item \textbf{Sensor}: Sistema 10–20 clínico (32–64 canales) {\tiny\cite{Schirrmeister2017}}
  \item \textbf{Metodología}:
    \begin{itemize}
      \item Clásicos: PSD, PLV/PLI + SVM/LDA, curvas ROC {\tiny\cite{Kay1998}}
      \item IA: EEGNet y Shallow ConvNet; opcional GCN sobre conectividad {\tiny\cite{Lawhern2018,Ajra2023}}
    \end{itemize}
  \item \textbf{Validación}: k-fold por sujeto; métricas accuracy, F1, AUC, \(P_D\), \(P_{FA}\); mapas de saliencia {\tiny\cite{Kay1998,Lawhern2018}}
\end{itemize}
\end{frame}

% ---- Papers ----
% \section{Papers}
\begin{frame}{Papers on EEG mapping with AI}
\begin{itemize}
  \item \textbf{Lawhern et al. 2018 (EEGNet)}: Multi-paradigma BCI, AUC/kappa, compacto, sensible al preprocesamiento {\tiny\cite{Lawhern2018}}
  \item \textbf{Schirrmeister et al. 2017 (Shallow/DeepConvNet)}: BCI IV-2a, intra- y cross-subject, baseline reproducible, necesita ajuste {\tiny\cite{Schirrmeister2017}}
  \item \textbf{Ajra et al. 2023 (Demencia + conectividad)}: EEG clínico AD/FTD vs HC, AEC/PLV, alta exactitud, falta vínculo con tau {\tiny\cite{Ajra2023}}
\end{itemize}
\end{frame}

% --- Datasets ---
% \section{Datasets}
\begin{frame}{Datasets}
  \begin{itemize}
  \item EEG clínico CN/MCI/AD (32–64 canales, sistema 10–20), organizado en BIDS {\tiny\cite{gil2023dataset, schirrmeister2017deep}}.
  \item Subcohorte con PET/LCR tau (si disponible); metadatos MMSE/MoCA {\tiny\cite{depaula2009tau, ajra2023deep}}.
  \item Dataset BCI IV-2a para validación técnica inicial {\tiny\cite{blankertz2008bci}}.
  \item 88 sujetos: CN, AD, FTD; balance clínico documentado {\tiny\cite{gil2023dataset}}.
  \item Validación cruzada por sujeto; segmentación en ventanas de 2 s {\tiny\cite{lawhern2018eegnet, schirrmeister2017deep}}.
\end{itemize}
\end{frame}

% --- Methodology (horizontal, compacto y con colores) ---
\begin{frame}{Methodology}
\centering
\vspace{-0.5cm} % Ajusta verticalmente si lo necesitas
\resizebox{0.90\textwidth}{!}{%
\begin{tikzpicture}[node distance=2cm, auto]

% Estilos por fase con colores
\tikzstyle{datosblock} = [rectangle, draw, rounded corners, fill=blue!20,
                          text centered, align=center,
                          minimum height=1.2cm, minimum width=3.5cm]
\tikzstyle{preprocblock} = [rectangle, draw, rounded corners, fill=gray!20,
                            text centered, align=center,
                            minimum height=1.2cm, minimum width=3.5cm]
\tikzstyle{tecblock} = [rectangle, draw, rounded corners, fill=green!20,
                        text centered, align=center,
                        minimum height=1.2cm, minimum width=3.5cm]
\tikzstyle{clinblock} = [rectangle, draw, rounded corners, fill=red!20,
                         text centered, align=center,
                         minimum height=1.2cm, minimum width=3.5cm]
\tikzstyle{evalblock} = [rectangle, draw, rounded corners, fill=orange!20,
                         text centered, align=center,
                         minimum height=1.2cm, minimum width=4.5cm]

% Nodos en horizontal
\node[datosblock] (datos) {
  \textbf{Datos}\\
  EEG CN/MCI/AD\\
  Sistema 10--20, 32--64 canales\\
  Formato BIDS
};

\node[preprocblock, right=of datos] (preproc) {
  \textbf{Preprocesamiento}\\
  Ventanas de 2 s\\
  Filtro 1--40 Hz\\
  Normalización trial-wise
};

\node[tecblock, right=of preproc, yshift=3cm] (faseTec) {
  \textbf{Fase técnica}\\
  A01--A09, 50 trials\\
  EEGNet (dropout 0.25, LR 0.001)\\
  ShallowConvNet (dropout 0.5)
};

\node[clinblock, right=of preproc, yshift=-3cm] (faseClin) {
  \textbf{Fase clínica}\\
  Sujeto CLINICO\\
  AMP, batch 64–512, LR 0.0005\\
  Tiempos hasta 2084 s
};

\node[evalblock, right=10cm of preproc] (eval) {
  \textbf{Evaluación}\\
  Accuracy, F1 macro, AUC ROC\\
  \( P_D \), \( P_{FA} \)\\
  Mapas por canal/banda\\
  Pérdida entrenamiento/validación\\
  Hiperparámetros por sujeto\\
  Observaciones por ejecución
};

% Conexiones
\draw[->, thick] (datos) -- (preproc);
\draw[->, thick] (preproc.east) -- (eval.west);
\draw[->, thick] (preproc) -- (faseTec.west);
\draw[->, thick] (preproc) -- (faseClin.west);
\draw[->, thick] (faseTec.east) -- (eval.north west);
\draw[->, thick] (faseClin.east) -- (eval.south west);

\end{tikzpicture}
}

\vspace{0.1cm}
\small
\begin{block}{Explicación}
\begin{itemize}
  \item Se parte de datos EEG clínicos (CN/MCI/AD) en formato BIDS.
  \item Las señales se preprocesan con segmentación, filtrado y normalización.
  \item Se entrenan modelos compactos (EEGNet y ShallowConvNet) en fases técnica y clínica.
  \item La evaluación incluye métricas (Accuracy, F1, AUC), pérdidas y observaciones por sujeto.
\end{itemize}
\end{block}

\end{frame}

% --- Results ---
\section{Results}
\begin{frame}{Resultados por sujeto y modelo}
\scriptsize
\renewcommand{\arraystretch}{1}
\vspace{-0.5cm}
\begin{table}[htbp]
\centering
\resizebox{\textwidth}{!}{%
\begin{tabular}{|c|c|c|c|p{5cm}|}
\hline
\textbf{Sujeto} & \textbf{Modelo} & \textbf{Accuracy (\%)} & \textbf{Val Loss} & \textbf{Observaciones} \\
\hline
A01 & \cellcolor{blue!30}EEGNet & 72.73 & 0.8028 & Segmentación automática (Mejor) \\
A01 & \cellcolor{green!15}ShallowConvNet & 45.45 & 1.4192 & Segmentación automática \\
A02 & \cellcolor{blue!30}EEGNet & 92.31 & 0.4277 & Segmentación automática (Mejor) \\
A02 & \cellcolor{green!15}ShallowConvNet & 76.92 & 1.7499 & Segmentación automática \\
A03 & \cellcolor{blue!30}EEGNet & 84.62 & 0.7852 & Segmentación automática (Mejor: menor pérdida) \\
A03 & \cellcolor{green!15}ShallowConvNet & 84.62 & 1.3167 & Segmentación automática \\
A04 & \cellcolor{blue!30}EEGNet & 93.75 & 0.4424 & Segmentación automática (Mejor) \\
A04 & \cellcolor{green!15}ShallowConvNet & 75.00 & 2.6222 & Segmentación automática \\
A05 & \cellcolor{blue!30}EEGNet & 90.62 & 0.4063 & Segmentación automática (Mejor) \\
A05 & \cellcolor{green!15}ShallowConvNet & 81.25 & 0.8011 & Segmentación automática \\
A06 & \cellcolor{blue!30}EEGNet & 93.75 & 0.3187 & Segmentación automática (Mejor) \\
A06 & \cellcolor{green!15}ShallowConvNet & 62.50 & 1.9865 & Segmentación automática \\
A07 & \cellcolor{blue!30}EEGNet & 87.50 & 0.5173 & Segmentación automática (Mejor) \\
A07 & \cellcolor{green!15}ShallowConvNet & 58.33 & 1.4671 & Segmentación automática \\
A08 & \cellcolor{blue!30}EEGNet & 93.75 & 0.2747 & Segmentación automática (Mejor) \\
A08 & \cellcolor{green!15}ShallowConvNet & 71.88 & 1.6626 & Segmentación automática \\
A09 & \cellcolor{blue!30}EEGNet & 96.88 & 0.1830 & Segmentación automática (Mejor) \\
A09 & \cellcolor{green!15}ShallowConvNet & 50.00 & 1.0928 & Segmentación automática \\
CLINICO & \cellcolor{blue!30}EEGNet-Clinico & 97.31 & 0.0931 & AMP, batch 128, tiempos promedio: 306.91 s (Mejor) \\
CLINICO & \cellcolor{green!15}ShallowConvNet-Clinico & 91.66 & 0.2624 & AMP, batch 512, tiempos promedio: 297.72 s \\
\hline
\end{tabular}
}
\caption{\tiny Comparación de desempeño por sujeto y modelo. El mejor modelo por sujeto está resaltado con color más intenso.}
\end{table}
\end{frame}

% --- Results Heatmap ---
%\section{Results}
\begin{frame}{Resultados por sujeto y modelo (Heatmap Accuracy)}
\scriptsize
\renewcommand{\arraystretch}{1.1}
\vspace{-0.5cm}
\begin{table}[htbp]
\centering
\resizebox{\textwidth}{!}{%
\begin{tabular}{|c|c|c|c|p{5cm}|}
\hline
\textbf{Sujeto} & \textbf{Modelo} & \textbf{Accuracy (\%)} & \textbf{Val Loss} & \textbf{Observaciones} \\
\hline
A01 & EEGNet & \cellcolor{yellow!30}72.73 & 0.8028 & Segmentación automática \\
A01 & ShallowConvNet & \cellcolor{red!30}45.45 & 1.4192 & Segmentación automática \\
A02 & EEGNet & \cellcolor{green!30}92.31 & 0.4277 & Segmentación automática \\
A02 & ShallowConvNet & \cellcolor{yellow!30}76.92 & 1.7499 & Segmentación automática \\
A03 & EEGNet & \cellcolor{yellow!30}84.62 & 0.7852 & Segmentación automática \\
A03 & ShallowConvNet & \cellcolor{yellow!30}84.62 & 1.3167 & Segmentación automática \\
A04 & EEGNet & \cellcolor{green!30}93.75 & 0.4424 & Segmentación automática \\
A04 & ShallowConvNet & \cellcolor{yellow!30}75.00 & 2.6222 & Segmentación automática \\
A05 & EEGNet & \cellcolor{green!30}90.62 & 0.4063 & Segmentación automática \\
A05 & ShallowConvNet & \cellcolor{yellow!30}81.25 & 0.8011 & Segmentación automática \\
A06 & EEGNet & \cellcolor{green!30}93.75 & 0.3187 & Segmentación automática \\
A06 & ShallowConvNet & \cellcolor{red!30}62.50 & 1.9865 & Segmentación automática \\
A07 & EEGNet & \cellcolor{yellow!30}87.50 & 0.5173 & Segmentación automática \\
A07 & ShallowConvNet & \cellcolor{red!30}58.33 & 1.4671 & Segmentación automática \\
A08 & EEGNet & \cellcolor{green!30}93.75 & 0.2747 & Segmentación automática \\
A08 & ShallowConvNet & \cellcolor{yellow!30}71.88 & 1.6626 & Segmentación automática \\
A09 & EEGNet & \cellcolor{green!30}96.88 & 0.1830 & Segmentación automática \\
A09 & ShallowConvNet & \cellcolor{red!30}50.00 & 1.0928 & Segmentación automática \\
CLINICO & EEGNet-Clinico & \cellcolor{green!30}97.31 & 0.0931 & AMP, batch 128, tiempos promedio: 306.91 s \\
CLINICO & ShallowConvNet-Clinico & \cellcolor{green!30}91.66 & 0.2624 & AMP, batch 512, tiempos promedio: 297.72 s \\
\hline
\end{tabular}
}
\caption{\tiny Heatmap de Accuracy: verde >90\%, amarillo 70–90\%, rojo <70\%.}
\end{table}
\end{frame}



% --- Limitations ---
% \section{Limitations}
\begin{frame}{Limitations}
  \begin{itemize}
  \item Alta variabilidad inter-sujeto en EEG → afecta generalización de modelos entre sujetos.
  \item Dataset clínico reducido → limita robustez estadística y validación cruzada.
  \item Interpretabilidad limitada de CNN → se requieren mapas por canal/banda para análisis clínico.
  \item Desbalance de clases y pocas muestras por clase → afecta estratificación y estabilidad en validación.
  \item Sensibilidad de ShallowConvNet al ruido → sobreajuste evidente en varios sujetos.
  \item Ausencia de biomarcadores estructurales (PET/LCR) → limita validación cruzada funcional.
  \item Dependencia del preprocesamiento → segmentación y normalización influyen en rendimiento.
  \item Costo computacional en fase clínica → AMP requiere tuning y tiempos por época elevados.
\end{itemize}


\end{frame}

% --- Conclusions ---
\section{Conclusions}
\begin{frame}{Conclusions}
\small
\vspace{-0.2cm}
\begin{itemize}
  \item EEGNet y ShallowConvNet son candidatos viables para biomarcadores EEG funcionales en demencia.
  \item EEGNet mostró mayor robustez, precisión y estabilidad frente a ShallowConvNet en fases técnica y clínica.
  \item Se alcanzó Accuracy >90\% en 8 de 10 sujetos con EEGNet, incluyendo fase clínica con AMP.
  \item Los mapas por canal/banda permiten interpretar patrones funcionales vinculados a trayectorias tau.
  \item El pipeline es reproducible, segmentado y adaptable a cohortes clínicas reales.
  \item Las limitaciones actuales (variabilidad, tamaño, interpretabilidad) abren camino a integración multimodal con PET/LCR.
\end{itemize}
\vspace{0.3cm}
\scriptsize
\textit{Conclusión: los modelos compactos permiten detectar alteraciones funcionales en EEG clínico, con potencial como biomarcadores no invasivos vinculados a tau.}
\end{frame}


% --- Future work ---
% \section{Future work}
\begin{frame}{Future work}
  \begin{itemize}
  \item Extender el pipeline a más datasets clínicos con segmentación automática y validación por sujeto.
  \item Integrar conectividad funcional (PLV/PLI) y arquitecturas GCN para análisis de redes neuronales.
  \item Validar hallazgos funcionales con biomarcadores estructurales (PET tau, LCR tau) si están disponibles.
  \item Explorar modelos multimodales que combinen EEG + tau + metadatos clínicos (MMSE, MoCA).
  \item Optimizar entrenamiento clínico con AMP y batch tuning para escalabilidad en cohortes grandes.
  \item Publicar mapas de saliencia por canal/banda como herramienta de interpretabilidad clínica.
\end{itemize}

\end{frame}

% --- Academic products ---
\begin{frame}{Academic Products}
\small
\vspace{-0.2cm}
\begin{itemize}
  \item \textbf{Monografía } sobre biomarcadores EEG funcionales en demencia.
  \item \textbf{Repositorio GitHub} con código reproducible, segmentación automática, entrenamiento AMP y comparaciones por sujeto.\\
  \texttt{\href{https://github.com/ivansst773/EEGNet_ShallowConvNet_Monografia}{github.com/ivansst773/EEGNet\_ShallowConvNet\_Monografia}}
  \item \textbf{Presentación interactiva en Streamlit Viewer} con visualizaciones, métricas y mapas por canal/banda.\\
  \texttt{\href{https://streamlit.io/view/ivansst773/eegnet_shallowconvnet_monografia}{streamlit.io/view/...}}
  \item \textbf{Bitácora técnica} documentada por fecha, sujeto, modelo y configuración.\\
  \texttt{bitacora.md}
  \item \textbf{Diagramas metodológicos en TikZ} con color institucional y narrativa clara.
  \item \textbf{Resultados clínicos validados} con EEGNet y ShallowConvNet sobre cohortes CN/MCI/AD.
\end{itemize}
\vspace{0.3cm}
\scriptsize
\textit{Todos los productos están disponibles públicamente para consulta, validación y extensión futura.}
\end{frame}


% --- Acknowledgements ---
\begin{frame}{Acknowledgements}
\small
\vspace{-0.2cm}
\begin{itemize}
  \item Universidad Nacional de Colombia – Sede Manizales, por el respaldo académico y técnico.
  \item Grupo de investigación, por el acompañamiento metodológico.
  \item Profesor Andrés Marino Álvarez Meza, PhD, por su invaluable guía científica, clínica y computacional, así como por su constante apoyo en el ámbito académico y personal durante todo el proceso.
  \item Comunidad académica y compañeros de semillero, por el intercambio de ideas y validación de resultados.
  \item Familia, por el apoyo constante durante todas las etapas del proyecto.
\end{itemize}
\vspace{0.3cm}
\scriptsize
\textit{Este trabajo es resultado de un esfuerzo colectivo, técnico y humano, con impacto académico y clínico.}
\end{frame}


% --- Acknowledgements ---
\begin{frame}{Acknowledgements}
\centering
\vspace{1cm}
{\Huge \textbf{¡Gracias!}}

\vspace{1cm}
{\large \textit{Este trabajo representa un paso hacia biomarcadores EEG funcionales accesibles, reproducibles y clínicamente relevantes.}}

\vspace{0.5cm}
\textbf{Edgar Iván Calpa Cuacialpud} \\
Universidad Nacional de Colombia – Sede Manizales \\
\texttt{eicalpac@unal.edu.co}
\texttt{\href{https://github.com/ivansst773/EEGNet_ShallowConvNet_Monografia}{github.com/ivansst773/EEGNet\_ShallowConvNet\_Monografia}}
\end{frame}


% ---- References ----
\begin{frame}[allowframebreaks]{References}
  \printbibliography
\end{frame}

\end{document}
